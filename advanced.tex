\section{Advanced}

\subsection{Command line tricks}
\begin{center}
\begin{longtable}{l|l}
\verb!cat xx | gvim - -c "v/^\d\d\|^[3-9]/d"! & filter a stream\\
\verb!ls | gvim -! & edit a stream!\\
\verb!gvim ftp://www.somedomain.com/index.html! & uses netrw.vim\\
\verb!gvim -h! & help\\
\verb!gvim -o file1 file2! & open into a split\\
\verb!gvim -c "/main" joe.c & open joe.c !& jump to "main"\\
\verb!gvim -c "%s/ABC/DEF/ge | update" file1.c! & execute multiple command on a single file\\
\verb!gvim -c "argdo %s/ABC/DEF/ge | update" *.c! & execute multiple command on a group of files\\
\verb!gvim -c "argdo /begin/+1,/end/-1g/^/d | update" *.c! & remove blocks of text from a series of files\\
\verb!gvim -s "convert.vim" file.c! & automate editing of a file (ex commands in convert.vim)\\
\verb!gvim -u NONE -U NONE -N! & load vim without .vimrc and plugins (clean vim)\\
\verb!gvim -c 'normal ggdG"*p' c:/aaa/xp! & access paste buffer contents (put in a script/batch file)\\
\verb?gvim -c 's/^/\=@*/|hardcopy!|q!'? & print paste contents to default printer\\
\verb!gvim -d file1 file2! & vimdiff (compare differences)\\
\verb!dp! & "put" difference under cursor to other file\\
\verb!do! & "get" difference under cursor from other file\\
\verb!:grep somestring *.php! & internal grep creates a list of all matching files\\
\verb!:h grep! & use :cn(ext) :cp(rev) to navigate list
\end{longtable}
\end{center}

\subsection{External programs}
\begin{center}
\begin{longtable}{l|l}
 \verb?:r!ls.exe? & reads in output of ls\\
 \verb?!!date? & same thing (but replaces/filters current line)\\
 \verb?:%!sort -u? & sort unique content\\
 \verb?:'a,'b!sort -u? & as above\\
 \verb?!1} sort -u? & sorts paragraph (note normal mode!!)\\
 \verb?map <F9> :w:!c:/php/php.exe %? & run file through php\\
 \verb?map <F2> :w:!perl -c %? & run file through perl\\
 \verb?:runtime! syntax/2html.vim? & convert txt to html
 \end{longtable}
\end{center}

\subsection{Recording}
\begin{center}
\begin{longtable}{l|l}
\verb!qq! & record to q\\
\verb!q! & end recording\\
\verb!@q! & to execute\\
\verb!@@! & to repeat\\
\verb!5@@! & to repeat 5 times\\
\verb!"qp! & display contents of register q (normal mode)\\
\verb!<ctrl-R>q! & display contents of register q (insert mode)\\
\verb!"qdd! & put changed contacts back into q\\
\verb!@q! & execute recording/register q\\
\verb!nnoremap ] @l:wbd! & combining a recording with a map (to end up in command mode)
\end{longtable}
\end{center}

\subsubsection{Operating a Recording on a Visual BLOCK}

define recording/register
\begin{verbatim}
qq:s/ to/ from/g^Mq
\end{verbatim}

define Visual BLOCK
\begin{verbatim}
V
\end{verbatim}

hit : and the following appears
\begin{verbatim}
:'<,'>
\end{verbatim}

complete as follows
\begin{verbatim}
:'<,'>norm @q
\end{verbatim}

\subsection{Quick jumping between splits}

\begin{verbatim}
map <C-J> <C-W>j<C-W>_
map <C-K> <C-W>k<C-W>_
\end{verbatim}

\subsection{Visual mode basics}
\begin{center}
\begin{longtable}{l|l}
\verb!v! & enter visual mode\\
\verb!V! & visual mode whole line\\
\verb!<C-V>! & enter VISUAL BLOCK mode\\
\verb!gv! & reselect last visual area (ultra)\\
\verb!o! & navigate visual area\\
\verb!"*y! & yank visual area into paste buffer\\
\verb!V%! & visualise what you match\\
\verb!V}J! & join visual block (great)\\
\verb!V}gJ! & join visual block w/o adding spaces\\
\verb!0<C-V>10j2ld! & delete first 2 characters of 10 successive lines
\end{longtable}
\end{center}

\subsection{vimrc essentials}
\begin{center}
\begin{longtable}{l|l}
\verb!set incsearch! & jumps to search word as you type\\
\verb!set wildignore=*.o,*.obj,*.bak,*.exe! & tab complete now ignores these\\
\verb!set shiftwidth=3! & for shift/tabbing\\
\verb!set vb t_vb=".! & set silent (no beep!)\\
\verb!set browsedir=buffer! & make 'open directory' use current directory\\
\end{longtable}
\end{center}

\subsubsection{Launching IE}

\begin{verbatim}
:nmap ,f :update<CR>:silent !start c:\progra~1\intern~1\iexplore.exe file://%:p<CR>
:nmap ,i :update<CR>: !start c:\progra~1\intern~1\iexplore.exe <cWORD><CR>
\end{verbatim}

\subsubsection{FTPing from vim}

\begin{verbatim}
cmap ,r :Nread ftp://209.51.134.122/public_html/index.html
cmap ,w :Nwrite ftp://209.51.134.122/public_html/index.html
gvim ftp://www.somedomain.com/index.html # uses netrw.vim
\end{verbatim}

\subsubsection{Autocmd}
\begin{center}
\begin{longtable}{l|l}
\verb?autocmd bufenter *.tex map <F1> :!latex %? & programming keys depending on file type\\
\verb?autocmd bufenter *.tex map <F2> :!xdvi -hush %<.dvi&? & launch xdvi with current file dvi\\
\verb?autocmd BufRead * silent! %s/[\r \t]\+\$//? & automatically delete whitespace, trailing dos returns\\
\verb!autocmd BufEnter *.php :%s/[ \t\r]\+\$//e! & same but only for php files
\end{longtable}
\end{center}

\subsection{Conventional shifting and indenting}
\begin{center}
\begin{longtable}{l|l}
 \verb!:'a,'b>>! & conventional Shifting/Indenting\\
 \verb!:vnoremap < <gv! & visual shifting (builtin-repeat)\\
 \verb!:vnoremap > >gv! & visual shifting (builtin-repeat)\\
 \verb!>i{! & block shifting (magic)\\
 \verb!>a{!\\
 \verb!>%!\\
 \verb!<%!
\end{longtable}
\end{center}

\subsection{Pulling objects onto command/search line}
\begin{center}
\begin{longtable}{l|l}
\verb!<C-R><C-W>! & pull word under the cursor into a command line or search\\
\verb!<C-R><C-A>! & pull WORD under the cursor into a command line or search\\
\verb!<C-R>-! & pull small register (also insert mode)\\
\verb!<C-R>[0-9a-z]! & pull named registers (also insert mode)\\
\verb!<C-R>%! & pull file name (also \verb!#!) (also insert mode)\\
\verb!<C-R>=somevar! & pull contents of a variable (eg \verb!:let sray="ray[0-9]"!)
\end{longtable}
\end{center}

\subsection{Capturing output of current script}
\begin{center}
\begin{longtable}{l|l}
 \verb?:new | r!perl #? & opens new buffer,read other buffer\\
 \verb?:new! x.out | r!perl #? & same with named file\\
 \verb?:new+read!ls?\\
 \verb?:new +put q|%!sort? & create a new buffer, paste a register "q" into it, then sort new buffer
\end{longtable}
\end{center}

\subsection{Inserting DOS carriage returns}
\begin{center}
\begin{longtable}{l|l}
 \verb!:%s/$\<C-V><C-M>&/g! & that's what you type\\
 \verb!:%s/$\<C-Q><C-M>&/g! & for Win32\\
 \verb!:%s/$/\^M&/g! & what you'll see where \verb!^M! is ONE character\\
 \verb!:set list! & display "invisible characters"
\end{longtable}
\end{center}

\subsection{Perform an action on a particular file or file type}

\begin{verbatim}
autocmd VimEnter c:/intranet/note011.txt normal! ggVGg?
autocmd FileType *.pl exec('set fileformats=unix')
\end{verbatim}

Retrieving last command line command for copy \& pasting into text
\begin{verbatim}
i<c-r>:
\end{verbatim}

Retrieving last Search Command for copy \& pasting into text
\begin{verbatim}
i<c-r>/
\end{verbatim}

\subsection{Inserting line number}

\begin{verbatim}
:g/^/exec "s/^/".strpart(line(".")."    ", 0, 4)
:%s/^/\=strpart(line(".")."     ", 0, 5)
:%s/^/\=line('.'). ' '
\end{verbatim}

\subsection{Numbering lines}
\begin{center}
\begin{longtable}{l|l}
 \verb!:set number! & show line numbers\\
 \verb?:map <F12> :set number!<CR>? & map to toggle line numbers\\
 \verb?:%s/^/\=strpart(line('.')."     ",0,&ts)?\\
 \verb?:'a,'b!perl -pne 'BEGIN{$a=223} substr($_,2,0)=$a++'? & number lines starting from arbitrary number\\
 \verb!qqmnYP`n^Aq! & in recording q repeat with @q\\
 \verb?:.,$g/^\d/exe "normal! \<c-a>"? & increment existing numbers to end of file\\
\end{longtable}
\end{center}

\subsection{Advanced incrementing}

\begin{verbatim}
let g:I=0
function! INC(increment)
    let g:I =g:I + a:increment
    return g:I
end function
\end{verbatim}

\subsubsection{Create list starting from 223 incrementing by 5 between markers a,b}

\begin{verbatim}
:let I=223
:'a,'bs/^/\=INC(5)/
\end{verbatim}

\subsubsection{Create a map for INC}

\begin{verbatim}
cab viminc :let I=223 \| 'a,'bs/$/\=INC(5)/
\end{verbatim}

\subsection{Digraphs (non alpha-numerics)}
\begin{center}
\begin{longtable}{l|l}
\verb!:digraphs! & display table\\
\verb!:h dig! & help\\
\verb!i<C-K>e'! & enters \'{e}\\
\verb!i<C-V>233! & enters \'{e} (Unix)\\
\verb!i<C-Q>233! & enters \'{e} (Win32)\\
\verb!ga! & View hex value of any character\\
\verb!:%s/\[<C-V>128-<C-V>255\]//gi ! & where you have to type the Control-V\\
\verb!:%s/\[€-ÿ\]//gi! & Should see a black square \& a dotted y\\
\verb!:%s/\[<C-V>128-<C-V>255<C-V>01-<C-V>31\]//gi! & All pesky non-asciis\\
\verb!%:exec "norm /\[\\x00-\\x1f\\x80-\\xff\]/"! & same thing\\
\verb!yl/<C-R>"! &  Pull a non-ascii character onto search bar
\end{longtable}
\end{center}

\subsection{Complex vim}
\begin{center}
\begin{longtable}{l|l}
\verb!:%s/\<\(on\|off\)\>/\=strpart("offon", 3 * ("off" == submatch(0)), 3)/g! & swap two words\\
\verb!:vnoremap <C-X> <Esc>`."gvP"P! & swap two words
\end{longtable}
\end{center}

\subsection{Syntax highlighting}
\begin{center}
\begin{longtable}{l|l}
 \verb!:set syntax=perl! & force Syntax coloring for a file that has no extension .pl\\
 \verb!:set syntax off! & remove syntax coloring (useful for all sorts of reasons)\\
 \verb!:colorscheme blue! & change coloring scheme (any file in $\sim$vim/vim??/colors)\\
 \verb!vim:ft=html:! & force HTML Syntax highlighting by using a modeline\\
 \verb!:syn match DoubleSpace " "! & example of setting your own highlighting\\
 \verb!:hi def DoubleSpace guibg=#e0e0e0! & sets the editor background
\end{longtable}
\end{center}

\subsection{Preventions and security}
\begin{center}
\begin{longtable}{l|l}
 \verb!:set noma! & prevents modifications (non modifiable)\\
 \verb!:set ro! & protect a file from unintentional writes (Read Only)\\
 \verb!:X! & encryption (do not forget your key!)
\end{longtable}
\end{center}

\subsection[Taglist]
{Taglist \footnote{Script required: taglist.vim (\href{http://www.vim.org/scripts/script.php?script\_id=273}{http://www.vim.org/scripts/script.php?script\_id=273})}}

\begin{center}
\begin{longtable}{l|l}
\verb!:Tlist! & display tags (list of functions)\\
\verb!<C-]> ! & jump to function under cursor
\end{longtable}
\end{center}

\subsection{Folding}
\begin{center}
\begin{longtable}{l|l}
 \verb!zf}! & fold paragraph using motion\\
 \verb!v}zf! & fold paragraph using visual\\
 \verb!zf'a! & fold to mark\\
 \verb!zo! & open fold\\
 \verb!zc! & re-close fold
\end{longtable}
\end{center}

\subsection{Renaming files}
Rename files without leaving vim
\begin{verbatim}
:r! ls *.c
:%s/\(.*\).c/mv & \1.bla
:w !sh
:q!
\end{verbatim}

\subsection{Reproducing previous line word by word}

\begin{verbatim}
imap ] @@@<esc>hhkyWjl?@@@<cr>P/@@@<cr>3s
nmap ] i@@@<esc>hhkyWjl?@@@<cr>P/@@@<cr>3s
\end{verbatim}

\subsection[Reading MS-Word documents]
{Reading MS-Word documents \footnote{Program required: Antiword (http://www.winfield.demon.nl)}}

\begin{verbatim}
:autocmd BufReadPre *.doc set ro
:autocmd BufReadPre *.doc set hlsearch!
:autocmd BufReadPost *.doc %!antiword "%"
\end{verbatim}

\subsection{Random functions}

\subsubsection{Save word under cursor to a file}

\begin{verbatim}
function! SaveWord()
    normal yiw
    exe ':!echo '.@0.' >> word.txt'
endfunction
\end{verbatim}

\subsubsection{Delete duplicate lines}

\begin{verbatim}
function! Del()
    if getline(".") == getline(line(".") - 1)
        norm dd
    endif
endfunction
\end{verbatim}

\subsubsection{Columnise a CSV file for display}

\begin{verbatim}
:let width = 20
:let fill=' ' | while strlen(fill) < width | let fill=fill.fill | endwhile
:%s/\([^;]*\);\=/\=strpart(submatch(1).fill, 0, width)/ge
:%s/\s\+$//ge
\end{verbatim}

\subsubsection{Highlight a particular csv column}

\begin{verbatim}
function! CSVH(x)
    execute 'match Keyword /^\([^,]*,\){'.a:x.'}\zs[^,]*/'
    execute 'normal ^'.a:x.'f,'
endfunction
command! -nargs=1 Csv :call CSVH(<args>)
\end{verbatim}

Call with \verb!:Csv 5! to highlight fifth column

\subsection{Miscallaenous commands}
\begin{center}
\begin{longtable}{l|l}
 \verb!:scriptnames! & list all plugins, .vimrcs loaded (super)\\
 \verb!:verbose set history?! & reveals value of history and where set\\
 \verb!:function! & list functions\\
 \verb!:func SearchCompl! & List particular function
\end{longtable}
\end{center}

\subsection{Vim traps}
\begin{list}{}
    \item In regular expressions you must backslash + (match 1 or more)
    \item In regular expressions you must backslash | (or)
    \item In regular expressions you must backslash ( (group)
    \item In regular expressions you must backslash \{ (count)
\end{list}

\begin{center}
\begin{longtable}{l|l}
\verb!/fred\+/! & matches fred/freddy but not free\\
\verb!/\(fred\)\{2,3}/! & note what you have to break\\
\verb!/codes\(\n\|\s\)*where! & normal regexp\\
\verb!/\vcodes(\n|\s)*where! & very magic
\end{longtable}
\end{center}

\subsection{Help}
\begin{center}
\begin{longtable}{l|l}
 \verb!:h quickref! & vim quick reference sheet (ultra)\\
 \verb!:h tips! & vim's own tips help\\
 \verb!:h visual<C-D><TAB>! & obtain list of all visual help topics\\
 \verb!:h ctrl<C-D>! & list help of all control keys\\
 \verb!:helpg uganda! & grep help files use :cn, :cp to find next\\
 \verb!:h :r! & help for :ex command\\
 \verb!:h CTRL-R! & normal mode\\
 \verb!:h /\r! & what's \verb!\r! in a regexp (matches a <CR>)\\
 \verb!:h \\zs! & double up backslash to find \verb!\zs! in help\\
 \verb!:h i_CTRL-R! & help for say <C-R> in insert mode\\
 \verb!:h c_CTRL-R! & help for say <C-R> in command mode\\
 \verb!:h v_CTRL-V! & visual mode\\
 \verb!:h tutor! & vim tutor\\
 \verb!<C-[>, <C-T>! & Move back \& forth in help history\\
 \verb!gvim -h! & vim command line help\\
 \verb!:helptags /vim/vim64/doc! & rebuild all *.txt help files in /doc\\
 \verb!:help add-local-help! &
\end{longtable}
\end{center}
