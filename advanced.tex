\section{Advanced}

\subsection{Command line tricks}
\begin{center}
\begin{longtable}{l|l}
cat xx $|$ gvim - -c ``v/\^{}$\backslash$d$\backslash$d$\backslash$$|$\^{}[3-9]/d `` & filter a stream\\
ls $|$ gvim - & edit a stream!\\
gvim \url{ftp://www.somedomain.com/index.html} & uses netrw.vim\\
gvim -h & help\\
gvim -o file1 file2 & open into a split\\
gvim -c ``/main'' joe.c & open joe.c \& jump to ``main''\\
gvim -c ``\%s/ABC/DEF/ge $|$ update'' file1.c & execute multiple command on a single file\\
gvim -c ``argdo \%s/ABC/DEF/ge $|$ update'' *.c & execute multiple command on a group of files\\
gvim -c ``argdo /begin/+1,/end/-1g/\^{}/d $|$ update'' *.c & remove blocks of text from a series of files\\
gvim -s ``convert.vim'' file.c & automate editing of a file (ex commands in convert.vim)\\
gvim -u NONE -U NONE -N & load vim without .vimrc and plugins (clean vim)\\
gvim -c 'normal ggdG''*p' c:/aaa/xp & access paste buffer contents (put in a script/batch file)\\
gvim -c 's/\^{}/$\backslash$=@*/$|$hardcopy!$|$q!' & print paste contents to default printer\\
gvim -d file1 file2 & vimdiff (compare differences)\\
dp & ``put'' difference under cursor to other file\\
do & ``get'' difference under cursor from other file\\
:grep somestring *.php & internal grep creates a list of all matching files\\
:h grep & use :cn(ext) :cp(rev) to navigate list
\end{longtable}
\end{center}

\subsection{External programs}
\begin{center}
\begin{longtable}{l|l}
 :r!ls.exe & reads in output of ls\\
 !!date & same thing (but replaces/filters current line)\\
 :\%!sort -u & sort unique content\\
 :'a,'b!sort -u & as above\\
 !1\} sort -u & sorts paragraph (note normal mode!!)\\
 map $<$F9$>$ :w:!c:/php/php.exe \% & run file through php\\
 map $<$F2$>$ :w:!perl -c \% & run file through perl\\
 :runtime! syntax/2html.vim & convert txt to html
 \end{longtable}
\end{center}

\subsection{Recording}
\begin{center}
\begin{longtable}{l|l}
qq & record to q\\
q & end recording\\
@q & to execute\\
@@ & to repeat\\
5@@ & to repeat 5 times\\
``qp & display contents of register q (normal mode)\\
$<$ctrl-R$>$q & display contents of register q (insert mode)\\
''qdd & put changed contacts back into q\\
@q & execute recording/register q\\
nnoremap ] @l:wbd & combining a recording with a map (to end up in command mode)
\end{longtable}
\end{center}

 \subsubsection{Operating a Recording on a Visual BLOCK}

 \begin{enumerate}
 \item define recording/register

 \begin{list}{}
    \item qq:s/ to/ from/g\^{}Mq\
 \end{list}

 \item Define Visual BLOCK
 \begin{list}{}
    \item V\}
 \end{list}

 \item hit : and the following appears
 \begin{list}{}
    \item :'$<$,'$>$
 \end{list}

 \item Complete as follows
 \begin{list}{}
    \item :'$<$,'$>$norm @q
 \end{list}

 \end{enumerate}

\subsection{Quick jumping between splits}
map $<$C-J$>$ $<$C-W$>$j$<$C-W$>$\_ \\
map $<$C-K$>$ $<$C-W$>$k$<$C-W$>$\_

\subsection{Visual mode basics}
\begin{center}
\begin{longtable}{l|l}
v & enter visual mode\\
V & visual mode whole line\\
$<$C-V$>$ & enter VISUAL BLOCK mode\\
gv & reselect last visual area (ultra)\\
o & navigate visual area\\
``*y & yank visual area into paste buffer\\
V\% & visualise what you match\\
V\}J & join visual block (great)\\
V\}gJ & join visual block w/o adding spaces\\
0$<$C-V$>$10j2ld & delete first 2 characters of 10 successive lines
\end{longtable}
\end{center}

\subsection{vimrc essentials}
\begin{center}
\begin{longtable}{l|l}
set incsearch & jumps to search word as you type\\
set wildignore=*.o,*.obj,*.bak,*.exe & tab complete now ignores these\\
set shiftwidth=3 & for shift/tabbing\\
set vb t\_vb=''. & set silent (no beep!)\\
set browsedir=buffer & make 'open directory' use current directory\\
\end{longtable}
\end{center}

\subsubsection{Launching IE}
nmap ,f :update:silent !start c:$\backslash$progra$\sim$1$\backslash$intern$\sim$1$\backslash$iexplore.exe \url{file://}\%:p\\
nmap ,i :update: !start c:$\backslash$progra$\sim$1$\backslash$intern$\sim$1$\backslash$iexplore.exe

\subsubsection{Ftping from vim}
cmap ,r :Nread \url{ftp://209.51.134.122/public}\_html/index.html\\
cmap ,w :Nwrite \url{ftp://209.51.134.122/public}\_html/index.html

\subsubsection{Autocmd}
\begin{center}
\begin{longtable}{l|l}
autocmd bufenter *.tex map $<$F1$>$ :!latex \% & programming keys depending on file type\\
autocmd bufenter *.tex map $<$F2$>$ :!xdvi -hush \%$<$.dvi\&  & launch xdvi with current file dvi\\
autocmd BufRead * silent! \%s/[$\backslash$r $\backslash$t]$\backslash$+\$// & automatically delete whitespace, trailing dos returns\\
autocmd BufEnter *.php :\%s/[ $\backslash$t$\backslash$r]$\backslash$+\$//e & same but only for php files
\end{longtable}
\end{center}

\subsection{Conventional shifting and indenting}
\begin{center}
\begin{longtable}{l|l}
 :'a,'b$>$$>$ & conventional Shifting/Indenting\\
 :vnoremap $<$ $<$gv & visual shifting (builtin-repeat)\\
 :vnoremap $>$ $>$gv & visual shifting (builtin-repeat)\\
 $>$i\{ & block shifting (magic)\\
 $>$a\{\\
 $>$\%\\
 $<$\%
\end{longtable}
\end{center}

\subsection{Pulling objects onto command/search line}
\begin{center}
\begin{longtable}{l|l}
$<$C-R$>$$<$C-W$>$ & pull word under the cursor into a command line or search\\
$<$C-R$>$$<$C-A$>$ & pull WORD under the cursor into a command line or search\\
$<$C-R$>$-                  & pull small register (also insert mode)\\
$<$C-R$>$[0-9a-z]           & pull named registers (also insert mode)\\
$<$C-R$>$\%                  & pull file name (also \#) (also insert mode)\\
$<$C-R$>$=somevar           & pull contents of a variable (eg :let sray="ray[0-9]")
\end{longtable}
\end{center}

\subsection{Capturing output of current script}
\begin{center}
\begin{longtable}{l|l}
 :new $|$ r!perl \# & opens new buffer,read other buffer\\
 :new! x.out $|$ r!perl \# & same with named file\\
 :new+read!ls\\
 :new +put q$|$\%!sort & create a new buffer, paste a register ``q'' into it, then sort new buffer
\end{longtable}
\end{center}

\subsection{Inserting DOS carriage returns}
\begin{center}
\begin{longtable}{l|l}
 :\%s/\$/$\backslash$\&/g & that's what you type\\
 :\%s/\$/$\backslash$\&/g & for Win32\\
 :\%s/\$/$\backslash$\^{}M\&/g & what you'll see where \^{}M is ONE character\\
 :set list & display ``invisible characters''
\end{longtable}
\end{center}

\subsection{Perform an action on a particular file or file type}
autocmd VimEnter c:/intranet/note011.txt normal! ggVGg?\\
autocmd FileType *.pl exec('set fileformats=unix')\\

`` Retrieving last command line command for copy \& pasting into text\\
i:\\
`` Retrieving last Search Command for copy \& pasting into text\\
i/\

\subsection{Inserting line number}
 :g/\^{}/exec ``s/\^{}/''.strpart(line(``.'').'' ``, 0, 4)\\
 :\%s/\^{}/$\backslash$=strpart(line(``.'').'' ``, 0, 5)\\
 :\%s/\^{}/$\backslash$=line('.'). ' '

\subsection{Numbering lines}
\begin{center}
\begin{longtable}{l|l}
 :set number & show line numbers\\
 :map $<$F12$>$ :set number!$<$CR$>$ & map to toggle line numbers\\
 :\%s/\^{}/$\backslash$=strpart(line('.').'' ``,0,\&ts)\\
 :'a,'b!perl -pne 'BEGIN\{\$a=223\} substr(\$\_,2,0)=\$a++' & number lines starting from arbitrary number\\
 qqmnYP`n\^{}Aq & in recording q repeat with @q\\
 :.,\$g/\^{}$\backslash$d/exe ``normal! $\backslash$'' & increment existing numbers to end of file\\
 o23qqYpq40@q & generate a list of numbers 23-64
\end{longtable}
\end{center}

\subsection{Advanced incrementing}

\begin{code}

\begin{verbatim}
    let g:I=0
    function! INC(increment)
        let g:I =g:I + a:increment
        return g:I
     end function
\end{verbatim}
\end{code}

\subsubsection{Create list starting from 223 incrementing by 5 between markers a,b}
:let I=223\\
:'a,'bs/\^{}/$\backslash$=INC(5)/

\subsubsection{Create a map for INC}
cab viminc :let I=223 $\backslash$$|$ 'a,'bs/\$/$\backslash$=INC(5)/

\subsection{Digraphs (non alpha-numerics)}
\begin{center}
\begin{longtable}{l|l}
:digraphs                         & display table\\
:h dig                            & help\\
i$<$C-K$>$e'                      & enters é\\
i$<$C-V$>$233                     & enters é (Unix)\\
i$<$C-Q$>$233                     & enters é (Win32)\\
ga                                & View hex value of any character\\
:%s/\[$<$C-V$>$128-$<$C-V$>$255\]//gi       & where you have to type the Control-V\\
:%s/\[€-ÿ\]//gi                     & Should see a black square & a dotted y\\
:%s/\[$<$C-V$>$128-$<$C-V$>$255$<$C-V$>$01-$<$C-V$>$31\]//gi & All pesky non-asciis\\
%:exec "norm /\[\\x00-\\x1f\\x80-\\xff\]/"    & same thing\\
yl/$<$C-R$>$" &  Pull a non-ascii character onto search bar
\end{longtable}
\end{center}

\subsection{Complex vim}
\begin{center}
\begin{longtable}{l|l}
:\%s/$\backslash$$<$$\backslash$(on$\backslash$$|$off$\backslash$)$\backslash$$>$/$\backslash$=strpart(``offon'', 3 * (``off'' == submatch(0)), 3)/g & swap two words\\
:vnoremap $<$C-X$>$ $<$Esc$>$`.``gvP``P & swap two words
\end{longtable}
\end{center}

\subsection{Syntax highlighting}
\begin{center}
\begin{longtable}{l|l}
 :set syntax=perl & force Syntax coloring for a file that has no extension .pl\\
 :set syntax off & remove syntax coloring (useful for all sorts of reasons)\\
 :colorscheme blue & change coloring scheme (any file in $\sim$vim/vim??/colors)\\
 vim:ft=html: & force HTML Syntax highlighting by using a modeline\\
 :syn match DoubleSpace `` `` & example of setting your own highlighting\\
 :hi def DoubleSpace guibg=\#e0e0e0 & sets the editor background
\end{longtable}
\end{center}

\subsection{Preventions and security}
\begin{center}
\begin{longtable}{l|l}
 :set noma (non modifiable) & prevents modifications\\
 :set ro (Read Only) & protect a file from unintentional writes\\
 :X & encryption (do not forget your key!)
\end{longtable}
\end{center}

\subsection[Taglist]
{Taglist \footnote{Script required: taglist.vim (\href{http://www.vim.org/scripts/script.php?script\_id=273}{http://www.vim.org/scripts/script.php?script\_id=273})}}

\begin{center}
\begin{longtable}{l|l}
:Tlist & display tags (list of functions)\\
$<$C-]$>$  & jump to function under cursor
\end{longtable}
\end{center}

\subsection{Folding}
\begin{center}
\begin{longtable}{l|l}
 zf\} & fold paragraph using motion\\
 v\}zf & fold paragraph using visual\\
 zf'a & fold to mark\\
 zo & open fold\\
 zc & re-close fold
\end{longtable}
\end{center}

\subsection{Renaming files}
 Rename files without leaving vim\\
 :r! ls *.c\\
 :\%s/$\backslash$(.*$\backslash$).c/mv \& $\backslash$1.bla\\
 :w !sh\\
 :q!

\subsection{Reproducing lines}
\begin{center}
\begin{longtable}{l|l}
 imap ] @@@hhkyWjl?@@@P/@@@3s & reproduce previous line word by word\\
 nmap ] i@@@hhkyWjl?@@@P/@@@3s & reproduce previous line word by word
\end{longtable}
\end{center}

\subsection[Reading MS-Word documents]
{Reading MS-Word documents \footnote{Program required: Antiword (http://www.winfield.demon.nl)}}

 :autocmd BufReadPre *.doc set ro\\
 :autocmd BufReadPre *.doc set hlsearch!\\
 :autocmd BufReadPost *.doc \%!antiword ``\%''

\subsection{Random functions}

\subsubsection{Save word under cursor to a file}

\begin{code}
\begin{verbatim}
    function! SaveWord()
        normal yiw
        exe ':!echo '.@0.' $>$$>$ word.txt'
    endfunction
\end{verbatim}
\end{code}

\subsubsection{Delete duplicate lines}

\begin{code}
\begin{verbatim}
    function! Del()
        if getline(``.'') == getline(line(``.'') - 1)
            norm dd
        endif
    endfunction
\end{verbatim}
\end{code}

\subsubsection{Columnise a CSV file for display}
:let width = 20\\
:let fill=' ' $|$ while strlen(fill) $<$ width $|$ let fill=fill.fill $|$ endwhile\\
:\%s/$\backslash$([\^{};]*$\backslash$);$\backslash$=/$\backslash$=strpart(submatch(1).fill, 0, width)/ge\\
:\%s/$\backslash$s$\backslash$+\$//ge

\begin{code}
\begin{verbatim}
    function! CSVH(x)
        execute 'match Keyword /\^{}$\backslash$([\^{},]*,$\backslash$)$\backslash$\{'.a:x.'\}$\backslash$zs[\^{},]*/'
        execute 'normal \^{}'.a:x.'f,'
    endfunction
\end{verbatim}
\end{code}

command! -nargs=1 Csv :call CSVH()\\
:Csv 5 : highlight fifth column

\subsection{Miscallaenous commands}
\begin{center}
\begin{longtable}{l|l}
 :scriptnames & list all plugins, \_vimrcs loaded (super)\\
 :verbose set history? & reveals value of history and where set\\
 :function & list functions\\
 :func SearchCompl & List particular function
\end{longtable}
\end{center}

\subsection{Vim traps}
\begin{list}{}
    \item In regular expressions you must backslash + (match 1 or more)
    \item In regular expressions you must backslash $|$ (or)
    \item In regular expressions you must backslash ( (group)
    \item In regular expressions you must backslash \{ (count)
\end{list}

\begin{center}
\begin{longtable}{l|l}
/fred$\backslash$+/ & matches fred/freddy but not free\\
/$\backslash$(fred$\backslash$)$\backslash$\{2,3\}/ & note what you have to break\\
/codes$\backslash$($\backslash$n$\backslash$|$\backslash$s$\backslash$)*where & normal regexp\\
/$\backslash$vcodes($\backslash$n$|$$\backslash$s)*where & very magic
\end{longtable}
\end{center}

\subsection{Help}
\begin{center}
\begin{longtable}{l|l}
 :h quickref & vim quick reference sheet (ultra)\\
 :h tips & vim's own tips help\\
 :h visual$<$C-D$>$$<$TAB$>$ & obtain list of all visual help topics\\
 :h ctrl$<$C-D$>$ & list help of all control keys\\
 :helpg uganda & grep help files use :cn, :cp to find next\\
 :h :r & help for :ex command\\
 :h CTRL-R & normal mode\\
 :h /$\backslash$r & what's $\backslash$r in a regexp (matches a $<$CR$>$)\\
 :h $\backslash$$\backslash$zs & double up backslash to find $\backslash$zs in help\\
 :h i\_CTRL-R & help for say $<$C-R$>$ in insert mode\\
 :h c\_CTRL-R & help for say $<$C-R$>$ in command mode\\
 :h v\_CTRL-V & visual mode\\
 :h tutor & vim tutor\\
 $<$C-[$>$, $<$C-T$>$  & Move back \& forth in help history\\
 gvim -h & vim command line help\\
 :helptags /vim/vim64/doc & rebuild all *.txt help files in /doc\\
 :help add-local-help
\end{longtable}
\end{center}
