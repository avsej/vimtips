\section{File Manipulation}

\subsection{Exploring}
\begin{center}
\begin{longtable}{l|l}
 :Exp(lore) & file explorer (note: capital E)\\
 :Sex(plore) & file explorer in split window\\
 :ls & list of buffers\\
 :cd .. & move to parent directory\\
 :args & list of files\\
 :lcd \%:p:h & change to directory of current file\\
 :autocmd BufEnter * lcd \%:p:h & change to directory of current file automatically \footnote{Script required: bufexplorer.vim \url{http://www.vim.org/script.php?script}\_id=42} (put in \_vimrc)\\
 $\backslash$be & buffer explorer list of buffers\\
 $\backslash$bs & buffer explorer (split window)
\end{longtable}
\end{center}

\subsection{Opening files \& other tricks}
\begin{center}
\begin{longtable}{l|l}
 gf & open file name under cursor (SUPER)\\
 :nnoremap gF :view  & open file under cursor, create if necessary\\
 ga & display hex,ascii value of char under cursor\\
 ggVGg? & rot13 whole file\\
 ggg?G & rot13 whole file (quicker for large file)\\
 :8 $|$ normal VGg? & rot13 from line 8\\
 :normal 10GVGg? & rot13 from line 8\\
 , & increment, decrement number under cursor\\
 =5*5 & insert 25 into text (mini-calculator)\\
 :e main\_ & tab completes\\
 main\_ & include NAME of file in text (insert mode)
\end{longtable}
\end{center}

\subsection{Multiple files management}
\begin{center}
\begin{longtable}{l|l}
 :bn & goto next buffer\\
 :bp & goto previous buffer\\
 :wn & save file and move to next (super)\\
 :wp & save file and move to previous\\
 :bd & remove file from buffer list (super)\\
 :bun & buffer unload (remove window but not from list)\\
 :badd file.c & file from buffer list\\
 :b 3 & go to buffer 3\\
 :b main & go to buffer with main in name eg main.c (ultra)\\
 :sav php.html & save current file as php.html and "move" to php.html\\
 :sav! \%$<$.bak & save current file to alternative extension (old way)\\
 :sav! \%:r.cfm & save current file to alternative extension\\
 :sav \%:s/fred/joe/ & do a substitute on file name\\
 :sav \%:s/fred/joe/:r.bak2 & do a substitute on file name \& ext.\\
 :!mv \% \%:r.bak & rename current file (DOS use rename or del)\\
 :e! & return to unmodified file\\
 :w c:/aaa/\% & save file elsewhere\\
 :e \# & edit alternative file (also cntrl-\^{})\\
 :rew & return to beginning of edited files list (:args)\\
 :brew & buffer rewind\\
 :sp fred.txt & open fred.txt into a split\\
 :sball,:sb & split all buffers (super)\\
 :scrollbind & in each split window\\
 :map $<$F5$>$ :ls$<$CR$>$:e \# & pressing F5 lists all buffers, just type number\\
 :set hidden & allows to change buffer w/o saving current buffer
 \end{longtable}
\end{center}

\subsection{File-name manipulation}
\begin{center}
\begin{longtable}{l|l}
 :h filename-modifiers & help\\
 :w \% & write to current file name\\
 :w \%:r.cfm & change file extention to .cfm\\
 :!echo \%:p & full path \& file name\\
 :!echo \%:p:h & full path only\\
 $<$C-R$>$\% & insert filename (insert mode)\\
 "\%p & insert filename (normal mode)\\
 /$<$C-R$>$\% & search for file name in text
\end{longtable}
\end{center}

\subsection{Command over multiple files}
\begin{center}
\begin{longtable}{l|l}
 :argdo \%s/foo/bar/e & operate on all files in :args\\
 :bufdo \%s/foo/bar/e\\
 :windo \%s/foo/bar/e\\
 :argdo exe '\%!sort'$|$w! & include an external command
\end{longtable}
\end{center}

\subsection{Sessions (set of files)}
\begin{center}
\begin{longtable}{l|l}
 gvim file1.c file2.c lib/lib.h lib/lib2.h & load files for "session"\\
 :mksession & create a session file (default session.vim)\\
 gvim -S Session.vim & reload all files
\end{longtable}
\end{center}

\subsection{Modelines}
\begin{center}
\begin{longtable}{l|l}
 vim:noai:ts=2:sw=4:readonly: & makes readonly\\
 vim:ft=html: & says use HTML syntax highlighting\\
 :h modeline & help with modelines
\end{longtable}
\end{center}

\subsubsection{Creating your own GUI Toolbar entry}

\begin{verbatim}
amenu  Modeline.Insert\ a\ VIM\ modeline
       \ <esc><esc>ggOvim:ff=unix ts=4 ss=4<CR>vim60:fdm=marker<esc>gg
\end{verbatim}

\subsection{Markers \& moving about}
\begin{center}
\begin{longtable}{l|l}
 '. & jump to last modification line (SUPER)\\
 `. & jump to exact spot in last modification line\\
 g; & cycle through recent changes (oldest first) \footnote{(new in vim 6.3)}\\
 g, & reverse direction \footnote{(new in vim 6.3)}\\
 :changes & show entire list of changes\\
 :h changelist & help for above\\
 $<$C-O$>$ & retrace your movements in file (starting from most recent)\\
 $<$C-I$>$ & retrace your movements in file (reverse direction)\\
 :ju(mps) & list of your movements\\
 :help jump-motions & explains jump motions\\
 :history & list of all your commands\\
 :his c & commandline history\\
 :his s & search history\\
 q/ & search history window (puts you in full edit mode)\\
 q: & commandline history window (puts you in full edit mode)\\
 : & history Window
 \end{longtable}
\end{center}

\subsection{Editing/moving within insert mode}
\begin{center}
\begin{longtable}{l|l}
$<$C-U$>$                             & delete all entered\\
$<$C-W$>$                             & delete last word\\
$<$HOME$>$$<$END$>$                   & beginning/end of line\\
$<$C-LEFTARROW$>$$<$C-RIGHTARROW$>$   & jump one word backwards/forwards\\
$<$C-X$>$$<$C-E$>$,$<$C-X$>$$<$C-Y$>$  & scroll while staying put in insert
\end{longtable}
\end{center}

\subsection{Abbreviations \& maps}
\begin{center}
\begin{longtable}{l|l}
:map $<$f7$>$   :'a,'bw! c:/aaa/x\\
:map $<$f8$>$   :r c:/aaa/x\\
:map $<$f11$>$  :.w! c:/aaa/xr$<$CR$>$\\
:map $<$f12$>$  :r c:/aaa/xr$<$CR$>$\\
:ab php & list of abbreviations beginning php\\
:map , & list of maps beginning ,\\
set wak=no & allow use of F10 for win32 mapping (:h winaltkeys)\\
$<$CR$>$             & enter\\
$<$ESC$>$            & escape\\
$<$BACKSPACE$>$      & backspace\\
$<$LEADER$>$         & backslash\\
$<$BAR$>$            & $|$\\
$<$SILENT$>$         & execute quietly\\
iab phpdb exit("$<$hr$>$Debug $<$C-R$>$a  "); & yank all variables into register a
\end{longtable}
\end{center}

\subsubsection{Display RGB colour under the cursor eg \#445588}

\begin{verbatim}
:nmap <leader>c :hi Normal guibg=#<c-r>=expand("<cword>")<cr><cr>
\end{verbatim}
